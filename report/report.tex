\documentclass{article} % For LaTeX2e
\usepackage{nips15submit_e,times}
\usepackage{hyperref}
\usepackage{url}
\usepackage{amsmath}
%\documentstyle[nips14submit_09,times,art10]{article} % For LaTeX 2.09


\title{Implementation and Analysis of Random Forests}


\author{
Jae Lee\\
School of Computing Science\\
Simon Fraser University\\
Burnaby BC V5A 1S6 \\
%\texttt{email} \\
\And
Richard Mar \\
School of Computing Science\\
Simon Fraser University\\
Burnaby BC V5A 1S6 \\
%\texttt{email} \\
\AND
Robin White \\
School of Mechatronic Systems Engineering\\
Simon Fraser University\\
Surrey BC V3T 0A3 \\
%\texttt{email} \\
}

\newcommand{\fix}{\marginpar{FIX}}
\newcommand{\new}{\marginpar{NEW}}

\nipsfinalcopy % Uncomment for camera-ready version

\begin{document}


\maketitle

\begin{abstract}
TODO: Decide if we need this section.
\end{abstract}

\section{Introduction}

\subsection{Decision Trees}
TODO: Adjust titles as necessary and add content specifying other papers.

\subsection{Ensemble Learning}
TODO: Adjust titles as necessary and add content specifying other papers.

\subsection{Random Forests}
TODO: Adjust titles as necessary and add content specifying other papers.

\section{Approach}
TODO: Figure out what is supposed to be here.

\section{Experiments}
TODO: Add content.

\subsection{Forest Size}
TODO: Add content.

\subsection{Tree Depth}
TODO: Add content.

\subsection{Splitting Criteria}

\textbf{Random Subspace Method}

Random Forest uses a modified splitting algorithm that attempts to further reduce correlation between individual trees. For example, if few attributes are strong predictors of the target label, these attributes will be selected in many trees leading to high correlation and greater generalization error.  Generalization error of an ensemble converges to the following expression:

\[Generalization\ error \leq \frac{corr(1-s^2)}{s^2}\] where corr is the average correlation among the trees and s is the average performance of individual classifiers. Thus, reducing correlation among the individual trees will also lower the generalization error.

 Work by Ho has demonstrated that average tree agreement or between trees is significantly lowered using the Random subspace method. [3] 
Estimating tree agreement between trees i and j as s\textsubscript{i,j}

\[ s_{i,j} = \frac{1}{n}\sum_{k=1}^{n}f(x_k)\]
\
\[ where \ f(x_k) = \begin{cases} 
      1 & if class\ decision_i(x_k) = class \ decision_j(x_k) \\
      0 & otherwise
   \end{cases}
\]

Ho's reults showed average of s\textsubscript{i,j} of random subspaces method was lower than that of bootstrapping and boosting methods alone. [3]

 Thus, limiting a tree's evaluation to only a subset of the actual feature set and  randomizing this subset during the tree's splitting  process helps to reduce correlation among each trees. The modified splitting algorithm will then split on a single feature with the best information gain ratio.

\subsubsection{Entropy}
TODO: Add content.

\subsubsection{Gini Index}
TODO: Add content.

\subsection{Custom Improvement?}



\begin{figure}[h]
\begin{center}
%\framebox[4.0in]{$\;$}
\fbox{\rule[-.5cm]{0cm}{4cm} \rule[-.5cm]{4cm}{0cm}}
\end{center}
\caption{TODO: Use this figure as a template for ours.}
\end{figure}

Table~\ref{sample-table}.

\begin{table}[t]
\caption{TODO: Use this table as a template for ours.}
\label{sample-table}
\begin{center}
\begin{tabular}{ll}
\multicolumn{1}{c}{\bf Machine Learning Package} &\multicolumn{1}{c}{\bf DESCRIPTION}
\\ \hline \\
Weka         &compare speed and accuracy at least \\
scikit-learn             &compare speed and accuracy at least \\
Tensorflow(?) Probably need one more, but not sure which one yet.             &compare speed and accuracy at least \\
\end{tabular}
\end{center}
\end{table}

\section{Conclusion}
TODO: Add content.

\subsubsection*{Contributions}
See GitLab project here for specific commits:\\
\href{
    https://csil-git1.cs.surrey.sfu.ca/rkm3/mlclass-1777-randomforest
}{
    https://csil-git1.cs.surrey.sfu.ca/rkm3/mlclass-1777-randomforest
}

\subsubsection*{References}


\small{
[1] Leo Breiman. 2001. Random Forests. Machine Learning. 45 1, 5-32.

[2]  J.A. Aslam, R.A. Popa, R.L. Rivest, "On Estimating the Size and Confidence of a Statistical Audit", Proc. Usenix/Accurate Electronic Voting Technology on Usenix/Accurate Electronic Voting Technology Workshop (EVT 07), pp. 8, 2007.

[3] T. K. Ho, "The random subspace method for constructing decision forests", IEEE Trans. Pattern Anal. Mach. Intell., vol. 20, no. 8, pp. 832-844, Aug. 1998.

TODO: Add more citations and fix formatting of existing one if it's incorrect.
}

\end{document}
